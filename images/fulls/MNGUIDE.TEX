% mnguide.tex
%
% v1.3 released 14th September 1995
% v1.2 released 5th September 1994 (M. Reed)
% v1.1 released 18th July 1994
% v1.0 released 28th January 1994

\documentstyle{mn}

% If your system has the AMS fonts version 2.0 installed, MN.sty can be
% made to use them by uncommenting the line: %\AMStwofontstrue
%
% By doing this, you will be able to obtain upright Greek characters.
% e.g. \umu, \upi etc.  See the section on "Upright Greek characters" in
% this guide for further information.
%
% If you are using AMS 2.0 fonts, bold math letters/symbols are available
% at a larger range of sizes for NFSS release 1 and 2 (using \boldmath or
% preferably \bmath).

\newif\ifAMStwofonts
%\AMStwofontstrue

%%%%% AUTHORS - PLACE YOUR OWN MACROS HERE %%%%%


%%%%%%%%%%%%%%%%%%%%%%%%%%%%%%%%%%%%%%%%%%%%%%%%

\ifoldfss
  %
  \newcommand{\rmn}[1] {{\rm #1}}
  \newcommand{\itl}[1] {{\it #1}}
  \newcommand{\bld}[1] {{\bf #1}}
  %
  \ifCUPmtlplainloaded \else
    \NewTextAlphabet{textbfit} {cmbxti10} {}
    \NewTextAlphabet{textbfss} {cmssbx10} {}
    \NewMathAlphabet{mathbfit} {cmbxti10} {} % for math mode
    \NewMathAlphabet{mathbfss} {cmssbx10} {} %  "   "    "
  \fi
  %
  \ifAMStwofonts
  %
    \ifCUPmtlplainloaded \else
      \NewSymbolFont{upmath} {eurm10}
      \NewSymbolFont{AMSa} {msam10}
      \NewMathSymbol{\upi}     {0}{upmath}{19}
      \NewMathSymbol{\umu}     {0}{upmath}{16}
      \NewMathSymbol{\upartial}{0}{upmath}{40}
      \NewMathSymbol{\leqslant}{3}{AMSa}{36}
      \NewMathSymbol{\geqslant}{3}{AMSa}{3E}
      \let\oldle=\le     \let\oldleq=\leq
      \let\oldge=\ge     \let\oldgeq=\geq
      \let\leq=\leqslant \let\le=\leqslant
      \let\geq=\geqslant \let\ge=\geqslant
    \fi
    %
  \fi
%
\fi % End of OFSS

\ifnfssone
  %
  \newmathalphabet{\mathit}
  \addtoversion{normal}{\mathit}{cmr}{m}{it}
  \addtoversion{bold}{\mathit}{cmr}{bx}{it}
  %
  \newcommand{\rmn}[1] {\mathrm{#1}}
  \newcommand{\itl}[1] {\mathit{#1}}
  \newcommand{\bld}[1] {\mathbf{#1}}
  %
  \def\textbfit{\protect\txtbfit}
  \def\textbfss{\protect\txtbfss}
  \long\def\txtbfit#1{{\fontfamily{cmr}\fontseries{bx}\fontshape{it}%
    \selectfont #1}}
  \long\def\txtbfss#1{{\fontfamily{cmss}\fontseries{bx}\fontshape{n}%
    \selectfont #1}}
  %
  \newmathalphabet{\mathbfit} % math mode version of \textbfit{..}
  \addtoversion{normal}{\mathbfit}{cmr}{bx}{it}
  \addtoversion{bold}{\mathbfit}{cmr}{bx}{it}
  %
  \newmathalphabet{\mathbfss} % math mode version of \textbfss{..}
  \addtoversion{normal}{\mathbfss}{cmss}{bx}{n}
  \addtoversion{bold}{\mathbfss}{cmss}{bx}{n}
  %
  \ifAMStwofonts
    %
    \ifCUPmtlplainloaded \else
      %
      % Make NFSS 1 use the extra sizes available for bold math italic and
      % bold math symbol. These definitions may already be loaded if your
      % NFSS format was built with fontdef.max.
      %
      \UseAMStwoboldmath
      %
      \makeatletter
      \new@mathgroup\upmath@group
      \define@mathgroup\mv@normal\upmath@group{eur}{m}{n}
      \define@mathgroup\mv@bold\upmath@group{eur}{b}{n}
      \edef\UPM{\hexnumber\upmath@group}
      %
      \new@mathgroup\amsa@group
      \define@mathgroup\mv@normal\amsa@group{msa}{m}{n}
      \define@mathgroup\mv@bold\amsa@group{msa}{m}{n}
      \edef\AMSa{\hexnumber\amsa@group}
      \makeatother
      %
      \mathchardef\upi="0\UPM19
      \mathchardef\umu="0\UPM16
      \mathchardef\upartial="0\UPM40
      \mathchardef\leqslant="3\AMSa36
      \mathchardef\geqslant="3\AMSa3E
      %
      \let\oldle=\le     \let\oldleq=\leq
      \let\oldge=\ge     \let\oldgeq=\geq
      \let\leq=\leqslant \let\le=\leqslant
      \let\geq=\geqslant \let\ge=\geqslant
      %
    \fi
  \fi
  %
\fi % End of NFSS release 1

\ifnfsstwo
  %
  \newcommand{\rmn}[1] {\mathrm{#1}}
  \newcommand{\itl}[1] {\mathit{#1}}
  \newcommand{\bld}[1] {\mathbf{#1}}
  %
  \def\textbfit{\protect\txtbfit}
  \def\textbfss{\protect\txtbfss}
  \long\def\txtbfit#1{{\fontfamily{cmr}\fontseries{bx}\fontshape{it}%
    \selectfont #1}}
  \long\def\txtbfss#1{{\fontfamily{cmss}\fontseries{bx}\fontshape{n}%
    \selectfont #1}}
  %
  \DeclareMathAlphabet{\mathbfit}{OT1}{cmr}{bx}{it}
  \SetMathAlphabet\mathbfit{bold}{OT1}{cmr}{bx}{it}
  \DeclareMathAlphabet{\mathbfss}{OT1}{cmss}{bx}{n}
  \SetMathAlphabet\mathbfss{bold}{OT1}{cmss}{bx}{n}
  %
  \ifAMStwofonts
    %
    \ifCUPmtlplainloaded \else
      \DeclareSymbolFont{UPM}{U}{eur}{m}{n}
      \SetSymbolFont{UPM}{bold}{U}{eur}{b}{n}
      \DeclareSymbolFont{AMSa}{U}{msa}{m}{n}
      \DeclareMathSymbol{\upi}{0}{UPM}{"19}
      \DeclareMathSymbol{\umu}{0}{UPM}{"16}
      \DeclareMathSymbol{\upartial}{0}{UPM}{"40}
      \DeclareMathSymbol{\leqslant}{3}{AMSa}{"36}
      \DeclareMathSymbol{\geqslant}{3}{AMSa}{"3E}
      %
      \let\oldle=\le     \let\oldleq=\leq
      \let\oldge=\ge     \let\oldgeq=\geq
      \let\leq=\leqslant \let\le=\leqslant
      \let\geq=\geqslant \let\ge=\geqslant
      %
    \fi
  \fi
  %
\fi % End of NFSS release 2

\ifCUPmtlplainloaded \else
  \ifAMStwofonts \else % If no AMS fonts
    \def\upi{\pi}
    \def\umu{\mu}
    \def\upartial{\partial}
  \fi
\fi

\long\def\boxit#1{\noindent\ignorespaces
  \framebox[\hsize][l]{\hbox{\vbox{\raggedright #1\par}}}\par
  \medskip\noindent\ignorespaces
} % for guide only


\title[Monthly Notices: \LaTeX\ guide for authors]
  {Monthly Notices of the Royal Astronomical 
  Society: \\ \LaTeX\ style guide for authors}
\author[A. Woollatt et al.]
  {A.~Woollatt,$^1$\thanks{Affiliated to ICRA.} 
  M.~Reed,$^1$ R.~Mulvey,$^1$ K.~Matthews,$^1$ 
  D.~Starling,$^1$ Y.~Yu,$^1$
  \newauthor % starts a new line in the 
             % author environment
  A.~Richardson$^1$ and P.~Smith$^2$\\
  $^1$Cambridge University Press, Shaftesbury
      Road, Cambridge CB2 2BS\\
  $^2$Blackwell Science,
      23 Ainslie Place, Edinburgh EH3 6AJ}
\date{Accepted 1993 December 11. Received 1993 March 17}
\pagerange{\pageref{firstpage}--\pageref{lastpage}}
\pubyear{1994}

\def\LaTeX{L\kern-.36em\raise.3ex\hbox{a}\kern-.15em
    T\kern-.1667em\lower.7ex\hbox{E}\kern-.125emX}

\newtheorem{theorem}{Theorem}[section]

\begin{document}

\label{firstpage}

\maketitle

\begin{abstract}
 This guide is for authors who are preparing papers for the
 Monthly Notices of the Royal Astronomical Society using the
\LaTeX\ document preparation system and the MN style file.
\end{abstract}

\begin{keywords}
 \LaTeX\ -- style files: \verb"mn.sty"\ -- sample text -- user guide.
\end{keywords}

\section{Introduction}

In addition to the standard submission of hardcopy from authors,
Monthly Notices accepts machine-readable forms of papers in \LaTeX.
The layout design for Monthly Notices has been implemented as a
\LaTeX\ style file. The MN style is based on the \verb"ARTICLE" style
as discussed in the \LaTeX\ manual \cite {la}. Commands which differ
from the standard \LaTeX\ interface, or which are provided in addition
to the standard interface, are explained in this guide. This guide is
not a substitute for the \LaTeX\ manual itself. Authors planning to
submit their papers in \LaTeX\ are advised to use \verb"mn.sty" as
early as possible in the creation of their files.

\subsection{The MN document style}

The use of \LaTeX\ document styles allows a simple change of style
(or style option) to transform the appearance of your document. The
MN style file preserves the standard \LaTeX\ interface such that any
document that can be produced using the standard \LaTeX\ \verb"ARTICLE"
style can also be produced with the MN style. However, the measure (or
width of text) is narrower than the default for \verb"ARTICLE", and even
narrower than for the \verb"A4" style, therefore line breaks will change
and long equations may need re-setting. 

When your article is printed in the Monthly Notices journal, it is typeset
in Monotype Times. As most authors do not have this font, it is likely that
the make-up will change with the change of font. For this reason, we ask
you to ignore details such as slightly long lines, page stretching, or
figures falling out of synchronization, because these details can be dealt with at a later stage.

\subsection{General style issues}

For general style issues, authors are referred to the `Instructions for
 Authors' on the inside back cover of Monthly Notices. Authors
who are interested in the details of style are referred to Butcher
\shortcite {bu} and The Chicago Manual \shortcite {ch}. The language of
the journal is British English and spelling should conform to this.

Use should be made of symbolic references (\verb"\ref") in order to
protect against late changes of order, etc.

\subsection{Submission of \LaTeX\ articles to the journal}

Authors are urged to state that electronic submission is intended when
first submitting their paper.
Papers should initially be submitted in the usual way to the Executive
Secretary, Royal Astronomical Society, Burlington House, London W1V 0NL,
as set out on the inside front cover of
each issue of Monthly Notices. Three hard copies should be supplied.
One of these copies should
be single-sided and double-spaced throughout, while the other two should
be weight-reduced, by being either single-spaced or double-sided. Note
that articles, or revised versions thereof, may not be submitted by
electronic mail. Appropriate gaps should be left for figures, original
versions and copies of which should also be supplied. Authors should
ensure that their figures are suitable (in terms of lettering size, etc.)
for the reductions they intend.

Authors who wish to include PostScript artwork dir\-ectly into their
articles can do so, by using either Tomas Rokicki's {\sf EPSF} macros
or Trevor Darrell's {\sf PSFIG} macros (both of which are supplied with
the DVIPS PostScript driver). Please remember to supply any additional
figure macros you use with your article.
Please also supply hard copies of your figures, for the use of production
editors and as a back-up.
Authors should not attempt to
use implementation specific \verb"\special"'s directly.

The correct Monthly Notices House Style should be used -- details are
given in the Instructions for Authors, in the Style Guide published
in the 1 January 1993 issue (\hbox{MNRAS}, 260, 1), and in Appendix~C of
this guide. No disk should be submitted
at this initial stage. When the paper has been accepted, the double-spaced
copy will be marked up by sub-editors at Blackwell Science
(BS) and returned to you for correction. When the requested corrections
have been made, you should send the following to BS:
%
\begin{enumerate}
\item a 3.5-inch or 5.25-inch PC disk (double- or high-density or
  Apple Mac), containing the {\em corrected\/} version of the paper,
  plus any macro files you have been using. The files for the final 
  version should be text-only, with no system-dependent control codes 
  (often called an ASCII file);
\item two hard copies of the corrected version, plus the original marked
  copy of the article; {\em any further author corrections should be
  clearly indicated on the latter\/};
\item a signed statement that the revised copy and the disk do indeed
  correspond.
\end{enumerate}
%
In addition, you should submit the following information concerning the
disk and its contents:
%
\begin{enumerate}
\item the disk format (e.g. IBM 360k);
\item how many files the disk contains, their names,
  a description of the file contents and the number of pages each file
  will produce when printed; details of any user-defined macros;
\item the computer system and implementation of \LaTeX\ used (e.g.
  IBM AT, PC\TeX\ v3.0).
\end{enumerate}
%
Ensure that any author-defined macros are gathered together in the file,
just before the \verb"\begin{document}" command. 

Unless the layout of the paper has to be significantly changed by the
typesetter (for example, if the gaps left for the figures are not
suitable), it will not be necessary to send you a typeset proof to check,
as the paper will be set directly from your submitted disk. However, you
should note that, if serious problems are encountered with the coding of
your paper (missing author-defined macros, for example), it may prove
necessary to divert the paper to conventional typesetting.

\section{Using the MN style}

If the file \verb"mn.sty" is not already in the appropriate system
directory for \LaTeX\ files, either arrange for it to be put there,
or copy it to your working directory. The MN document style is implemented
as a complete document style, {\em not\/} a document style option. In
order to use the MN style, replace \verb"article" by \verb"mn" in the
\verb"\documentstyle" command at the beginning of your document:
%
\begin{verbatim}
\documentstyle{article}
\end{verbatim}
%
is replaced by
%
\begin{verbatim}
\documentstyle{mn}
\end{verbatim}
%
In general, the following standard document style options should {\em
not\/} be used with the MN style:
%
\begin{enumerate}
  \item {\tt 10pt}, {\tt 11pt}, {\tt 12pt} -- unavailable;
  \item {\tt twoside} (no associated style file) -- {\tt
     twoside} is the default;
  \item {\tt fleqn}, {\tt leqno}, {\tt titlepage} --
        should not be used (\verb"fleqn" is already incorporated into
        the MN style);
  \item {\tt twocolumn} -- is not necessary as it is the default style.
\end{enumerate}
%
If necessary, {\tt draft}, {\tt proc}, {\tt ifthen} and {\tt bezier} can be used.

The MN style file has been designed to operate with the standard version
of \verb"lfonts.tex" that is distributed as part of \LaTeX. If you have
access to the source file for this guide, \verb"mnguide.tex", and to the
specimen article, \verb"mnsample.tex", attempt to typeset both of these.
If you find font problems you might investigate whether a non-standard
version of \verb"lfonts.tex" has been installed in your system.

\subsection{Additional document style options}

The following additional style options are available with the MN style:
\begin{description}
  \item {\tt onecolumn} -- to be used {\it only} when two-column output
        is unable to accommodate long equations;
  \item {\tt landscape} -- for producing wide figures and tables which
        need to be included in landscape format (i.e.\ sideways) rather
        than portrait (i.e.\ upright). This option is described below. 
  \item {\tt doublespacing} -- this will double-space your 
        article by setting \verb"\baselinestretch" to 2.
  \item {\tt referee} -- 12/20pt text size, single column,
        measure 16.45cm, left margin 2.75cm on A4 page.
  \item {\tt galley} -- no running heads, no attempt to align
        the bottom of columns.
\end{description}

\subsection{Landscape pages}

If a table or illustration is too wide to fit the standard measure, it
must be turned, with its caption, through 90 degrees anticlockwise. 
Landscape illustrations and/or tables cannot be produced directly
using the MN style file because \TeX\ itself cannot turn the
page, and not all device drivers provide such a facility.
The following procedure can be used to produce such pages.
%
\begin{enumerate}
  \item Use the \verb"table*" or \verb"figure*" environments in your
        document to create the space for your table or figure on the
        appropriate page of your document. Include an empty 
        caption in this environment to ensure the correct
        numbering of subsequent tables and figures. For instance, the 
        following code prints a page with the running head, a message
        half way down and the figure number towards the bottom. If you 
        are including a plate, the running headline is different, and you
        need to key in the three lines that are marked with \verb"% **",
        with an appropriate headline.
%
\begin{verbatim}
% ** \clearpage 
% ** \thispagestyle{plate}
% ** \plate{Opposite p.~812, MNRAS, {\bf 261}}
\begin{figure*}
  \vbox to220mm{\vfil Landscape figure to go here.
  \caption{}
 \vfil}
 \label{landfig}
\end{figure*}
\end{verbatim}
%
\item Create a separate document with the corresponding document style
      but also with the \verb"landscape" document style option, and 
      include the \verb"\pagestyle" command, as follows:
%
\begin{verbatim}
\documentstyle[landscape]{mn}
\pagestyle{empty}
\end{verbatim}
%
  \item Include your complete tables and illustrations (or space for
        these) with captions using the \verb"table*" and \verb"figure*"
        environments.  
  \item Before each float environment, use the 
        \verb"\setcounter" command to ensure the correct numbering of 
        the caption. For example,
%
\begin{verbatim}
\setcounter{table}{0}
\begin{table*}
 \begin{minipage}{115mm}
 \caption{The Largest Optical Telescopes.}
 \label{tab1}
 \begin{tabular}{@{}llllcll}
   :
 \end{tabular}
 \end{minipage}
\end{table*}
\end{verbatim}
%
The corresponding example for a figure would be:
%
\begin{verbatim}
\clearpage
\setcounter{figure}{12}
\begin{figure*}
 \vspace{144mm}
 \caption{Chart for a cold plasma.}
 \label{fig13}
\end{figure*}
\end{verbatim}
\end{enumerate}


\section{Additional facilities}

In addition to all the standard \LaTeX\ design elements, the MN style
includes the following features.
%
\begin{enumerate}
  \item Extended commands for specifying a short version of the title and
        author(s) for the running headlines.
  \item A \verb"keywords" environment and a \verb"\nokeywords" command.
  \item Use of the \verb"description" environment for unnumbered lists.
  \item A \verb"\contcaption" command to produce captions for continued
        figures or tables.
 \end{enumerate}
%
In general, once you have used the additional \verb"mn.sty" facilities
in your document, do not process it with a standard \LaTeX\ style file.

\subsection{Titles and author's name}

In the MN style, the title of the article and the author's name (or
authors' names) are used both at the beginning of the article for the
main title and throughout the article as running headlines at the top
of every page. The title is used on odd-numbered pages (rectos) and the
author's name appears on even-numbered pages (versos). Although the
main heading can run to several lines of text, the running headline
must be a single line ($\le 45$ characters). Moreover, the main
heading can also incorporate new line commands (e.g. \verb"\\") but
these are not acceptable in a running headline. To enable you to
specify an alternative short title and an alternative short author's
name, the standard \verb"\title" and \verb"\author" commands have been
extended to take an optional argument to be used as the running
headline. The running headlines for this guide were produced using the
following code:
%
\begin{verbatim}
\title[Monthly Notices: \LaTeX\ guide for authors]
  {Monthly Notices of the Royal Astronomical 
  Society: \\ \LaTeX\ style guide for authors}
\end{verbatim}
%
and
%
\begin{verbatim}
\author[A. Woollatt et al.]
  {A.~Woollatt,$^1$\thanks{Affiliated to ICRA.} 
  M.~Reed,$^1$ R.~Mulvey,$^1$ K.~Matthews,$^1$ 
  D.~Starling,$^1$ Y.~Yu,$^1$
  \newauthor % starts a new line in the 
             % author environment
  A.~Richardson$^1$ and P.~Smith$^2$\\
  $^1$Cambridge University Press, Shaftesbury
      Road, Cambridge CB2 2BS\\
  $^2$Blackwell Science,
      23 Ainslie Place, Edinburgh EH3 6AJ}
\end{verbatim}
%
The \verb"\thanks" note produces a footnote to the title or author.

\subsection{Key words and abstracts}

At the beginning of your article, the title should be generated in the
usual way using the \verb"\maketitle" command. Immediately following
the title you should include an abstract followed by a list of key
words. The abstract should be enclosed within an \verb"abstract"
environment, followed immediately by the key words enclosed in a
\verb"keywords" environment. For example, the titles for this guide
were produced by the following source:
%
\begin{verbatim}
\maketitle
\begin{abstract}
 This guide is for authors who are preparing
 papers for the Monthly Notices of the
 Royal  Astronomical Society using the \LaTeX\
 document preparation system and the MN style
 file.
\end{abstract}
\begin{keywords}
 \LaTeX\ -- style files: \verb"mn.sty"\ --
 sample text -- user guide.
\end{keywords}

\section{Introduction}
  :
\end{verbatim}
%
The heading `{\bf Key words}' is included automatically and the key
words are followed by vertical space. If, for any reason, there are no
key words, you should insert the \verb"\nokeywords" command immediately
after the end of the \verb"abstract" environment. This ensures that the
vertical space after the abstract and/or title is correct and that any
\verb"thanks" acknowledgments are correctly included at the bottom of
the first column. For example,
%
\begin{verbatim}
\maketitle
\begin{abstract}
  :
\end{abstract}
\nokeywords

\section{Introduction}
  :
\end{verbatim}

\subsection{Lists}

The MN style provides numbered lists using the \verb"enumerate"
environment and unnumbered lists using the \verb"description"
environment with an empty label. Bulleted lists are not part of the MN
style and the \verb"itemize" environment should not be used.

The enumerated list numbers each list item with roman numerals:
%
\begin{enumerate}
  \item first item
  \item second item
  \item third item
\end{enumerate}
%
Alternative numbering styles can be achieved by inserting a
redefinition of the number labelling command after the
\verb"\begin{enumerate}". For example, the list
%
\begin{enumerate}
\renewcommand{\theenumi}{(\arabic{enumi})}
  \item first item
  \item second item
  \item etc\ldots
\end{enumerate}
%
was produced by:
%
\begin{verbatim}
\begin{enumerate}
 \renewcommand{\theenumi}{(\arabic{enumi})}
  \item first item
       :
\end{enumerate}
\end{verbatim}
%
Unnumbered lists are provided using the \verb"description" environment.
For example,
\begin{description}
  \item First unnumbered item which has no label and is indented from
        the left margin.
  \item Second unnumbered item.
  \item Third unnumbered item.
\end{description}
was produced by:
%
\begin{verbatim}
\begin{description}
 \item First unnumbered item...
 \item Second unnumbered item.
 \item Third unnumbered item.
\end{description}
\end{verbatim}

\subsection{Captions for continued figures and tables}\label{contfigtab}

The \verb"\contcaption" command may be used to produce a caption with the
same number as the previous caption (for the corresponding type of
float). For instance, if a very large table does not fit on one page,
it must be split into two floats; the second float should use the
\verb"\contcaption" command:
%
\begin{verbatim}
\begin{table}
 \contcaption{}
  \begin{tabular}{@{}lccll}
  :
  \end{tabular}
\end{table}
\end{verbatim}


\section[]{Some guidelines for using\\* standard facilities}

The following notes may help you achieve the best effects with the MN
style file.

\subsection{Sections}

\LaTeX\ provides five levels of section headings and they are all
defined in the MN style file:
\begin{description}
  \item \verb"\section"
  \item \verb"\subsection"
  \item \verb"\subsubsection"
  \item \verb"\paragraph"
  \item \verb"\subparagraph"
\end{description}
Section numbers are given for section, subsection, subsubsection
and paragraph headings.  Section headings are automatically converted to
upper case; if you need any other style, see the example in Section~\ref{headings}.

If you find your section/subsection (etc.)\ headings are wrapping round, 
you must use the \verb"\\*" to end individual lines and include the 
optional argument \verb"[]" in the section command. This ensures that 
the turnover is flushleft.

\subsection{Illustrations (or figures)}

The MN style will cope with most positioning of your illustrations and
you should not normally use the optional positional qualifiers on the
\verb"figure" environment which would override these decisions. See
`Instructions for Authors' in Monthly Notices for submission of
artwork. Figure captions should be below the figure itself, therefore
the \verb"\caption" command should appear after the figure or space
left for an illustration. For example, Fig.~\ref{sample-figure} is
produced using the following commands:
%
\begin{verbatim}
\begin{figure}
 \vspace{5.5cm}
 \caption{An example figure in which space has
          been left for the artwork.}
 \label{sample-figure}
\end{figure}
\end{verbatim}

\begin{figure}
  \vspace{5.5cm}
  \caption{An example figure in which space has been
           left for the artwork.}
  \label{sample-figure}
\end{figure}

\subsection{Tables}

The MN style will cope with most positioning of your tables and you
should not normally use the optional positional qualifiers on the
\verb"table" environment which would override these decisions. Table
captions should be at the top, therefore the \verb"\caption" command
should appear before the body of the table. 

The \verb"tabular" environment can be used to produce tables with
single horizontal rules, which are allowed, if desired, at the head and
foot only. This environment has been modified for the MN style in the
following ways:
%
\begin{enumerate}
  \item additional vertical space is inserted on either side of a rule;
  \item vertical lines are not produced.
\end{enumerate}
%
Commands to redefine quantities such as \verb"\arraystretch" should be
omitted. For example, Table~\ref{symbols} is produced using the
following commands. Note that \verb"\rmn" will produce a roman character 
in math mode. It has been defined in two ways in the source code of
the guide, one way for authors using the New Font Selection Scheme, and 
the other for authors using the old font selection scheme. There are
also \verb"\bld" and \verb"\itl", which produce bold face and text italic
in math mode.
\begin{table}
 \caption{Radio-band beaming model parameters 
          for FSRQs and BL Lacs.}
 \label{symbols}
 \begin{tabular}{@{}lcccccc}
  Class & $\gamma _1$ & $\gamma _2$ 
        & $\langle \gamma \rangle$
        & $G$ & $f$ & $\theta _{\rmn{c}}$ \\
  BL Lacs &5 & 36 & 7 & $-4.0$ 
        & $1.0\times 10^{-2}$ & 10$^\circ$ \\
  FSRQs & 5 & 40 & 11 & $-2.3$ 
        & $0.5\times 10^{-2}$ & 14$^\circ$ \\
 \end{tabular}

 \medskip
 {\em G} is the slope of the Lorentz factor distribution, i.e.
 $n(\gamma)\propto \gamma ^G$, extending between $\gamma _1$ and
 $\gamma_2$, with mean value $\langle \gamma \rangle$, {\em f\/} is the
 ratio between the intrinsic jet luminosity and the extended, unbeamed
 luminosity, while $\theta_{\rmn{c}}$ is the critical angle separating
 the beamed class from the parent population.
\end{table}
\begin{verbatim}
\begin{table}
 \caption{Radio-band beaming model parameters 
          for FSRQs and BL Lacs.}
 \label{symbols}
 \begin{tabular}{@{}lcccccc}
  Class & $\gamma _1$ & $\gamma _2$ 
        & $\langle \gamma \rangle$
        & $G$ & $f$ & $\theta _{\rmn{c}}$ \\
  BL Lacs &5 & 36 & 7 & $-4.0$ 
        & $1.0\times 10^{-2}$ & 10$^\circ$ \\
  FSRQs & 5 & 40 & 11 & $-2.3$ 
        & $0.5\times 10^{-2}$ & 14$^\circ$ \\
 \end{tabular}

 \medskip
 {\em G} is the slope of the Lorentz factor 
   :
 class from the parent population.
\end{table}
\end{verbatim}
%
If you have a table that is to extend over two columns, you need to use
\verb"table*" in a minipage environment, i.e., you can say
%
\begin{verbatim}
\begin{table*}
\begin{minipage}{80mm}
 \caption{Caption which will wrap round to the
          width of the minipage environment.}
 \begin{tabular}{%
      :
 \end{tabular}
\end{minipage}
\end{table*}
\end{verbatim}
%
The width of the minipage should more or less be the width of your table,
so you can only guess on a value on the first pass. The value will have to 
be adjusted when your article is typeset in Times, so don't worry about
making it the exact size.

\subsection{Running headlines}

As described above, the title of the article and the author's name (or
authors' names) are used as running headlines at the top of every page.
The headline on left-hand pages can list up to three names; for more than
three use et~al. The \verb"\pagestyle" and \verb"\thispagestyle"
commands should {\em not\/} be used. Similarly, the commands
\verb"\markright" and \verb"\markboth" should not be necessary.


\subsection[]{Typesetting mathematics}\label{TMth}

\subsubsection{Displayed mathematics}

The MN style will set displayed mathematics flush with the left margin,
provided that you use the \LaTeX\ standard of open and closed square
brackets as delimiters. The equation
\[
 \sum_{i=1}^p \lambda_i = \rmn{trace}(\bld{S})
\]
was typeset in the MN style using the commands
%
\begin{verbatim}
\[
 \sum_{i=1}^p \lambda_i = \rmn{trace}(\bld{S})
\]
\end{verbatim}
%
Note the difference between the positioning of this equation and of
the following centred equation,
$$ \alpha_{j+1} > \bar{\alpha}+ks_{\alpha} $$
which was (wrongly) typeset using double dollars as follows:
%
\begin{verbatim}
$$ \alpha_{j+1} > \bar{\alpha}+ks_{\alpha} $$
\end{verbatim}


\subsubsection{Bold math italic / bold symbols}

To get bold math italic you can use \verb"\boldmath" (which only works for
the `normal' size), but in many implementations of \LaTeX\ this has not
been defined at 9pt. An alternative is to use \verb"\bmath" which works
for all sizes. e.g.
%
\begin{verbatim}
\[
  d(\bmath{s_{t_u}}) = \langle [RM(\bmath{X_y}
  + \bmath{s_t}) - RM(\bmath{x_y})]^2 \rangle
\]
\end{verbatim}
%
to produce:
\[
  d(\bmath{s_{t_u}}) = \langle [RM(\bmath{X_y}
  + \bmath{s_t}) - RM(\bmath{x_y})]^2 \rangle
\]
Working this way, scriptstyle and scriptscriptstyle sizes will take care of themselves.

\subsubsection{Bold Greek}\label{boldgreek}

Bold lowercase Greek characters can now be obtained by prefixing
the normal (unbold) symbol name with a `b', e.g.\ \verb"\bgamma" gives
$\bgamma$. This rule does not apply to bold \verb"\eta", as this would lead
to a name clash with \verb"\beta". Instead use \verb"\boldeta" for bold eta.
Note that there is no \verb"\omicron" (so there is no \verb"\bomicron"),
just use `o' in math mode for omicron ($o$) and `\verb"\bmath{o}"' for bold omicron ($\bmath{o}$).

For bold uppercase Greek, prefix the unbold character name with
\ifoldfss
%
\verb"\bf", e.g.\ \verb"\bf\Gamma" gives $\bf\Gamma$.
%
\else
%
\verb"\mathbf", e.g.\ \verb"\mathbf\Gamma" gives $\mathbf\Gamma$.
%
\fi
Upper and lowercase Greek characters are available in all typesizes.

You can then use these definitions in math mode, as you would normal Greek
characters:
%
\ifoldfss
%
\begin{verbatim}
\[
  \balpha_{\bmu} = {\bf\Theta} \alpha.
\]
\end{verbatim}
%
\else
%
\begin{verbatim}
\[
  \balpha_{\bmu} = \mathbf{\Theta} \alpha.
\]
\end{verbatim}
%
\fi
%
will produce
%
\ifoldfss
%
\[
  \balpha_{\bmu} = {\bf\Theta} \alpha.
\]
%
\else
%
\[
  \balpha_{\bmu} = \mathbf{\Theta} \alpha.
\]
%
\fi

\subsubsection{Upright Greek characters}\label{upgreek}

You can obtain upright Greek characters if you have access to the
American Maths Society Euler fonts (version 2.0), but you may not
have these. In this case, you will have to use the normal math italic
symbols and the typesetter will substitute the corresponding upright
characters. You will make this easier if you can use the macros \verb|\upi|,
\verb|\umu| and \verb|\upartial| etc.\ in your text to indicate the need for upright characters, together with the following definitions in the preamble
(before \verb|\begin{document}|):

\subsubsection*{Authors with AMS fonts}

\ifoldfss
%
\begin{verbatim}
\ifCUPmtlplainloaded \else
  \NewSymbolFont{upmath} {eurm10}
  \NewMathSymbol{\upi}     {0}{upmath}{19}
  \NewMathSymbol{\umu}     {0}{upmath}{16}
  \NewMathSymbol{\upartial}{0}{upmath}{40}
\fi
\end{verbatim}
%
\fi

\ifnfssone
%
\begin{verbatim}
\ifCUPmtlplainloaded \else
  \makeatletter
  \new@mathgroup\upmath@group
  \define@mathgroup\mv@normal\upmath@group{eur}{m}{n}
  \define@mathgroup\mv@bold\upmath@group{eur}{b}{n}
  \edef\UPM{\hexnumber\upmath@group}
  \makeatother
  \mathchardef\upi="0\UPM19
  \mathchardef\umu="0\UPM16
  \mathchardef\upartial="0\UPM40
\fi
\end{verbatim}
%
\fi

\ifnfsstwo
%
\begin{verbatim}
\ifCUPmtlplainloaded \else
  \DeclareSymbolFont{UPM}{U}{eur}{m}{n}
  \SetSymbolFont{UPM}{bold}{U}{eur}{b}{n}
  \DeclareMathSymbol{\upi}{0}{UPM}{"19}
  \DeclareMathSymbol{\umu}{0}{UPM}{"16}
  \DeclareMathSymbol{\upartial}{0}{UPM}{"40}
\fi
\end{verbatim}
%
\fi

\subsubsection*{Authors without AMS fonts}

\begin{verbatim}
\ifCUPmtlplainloaded \else
  \def\umu{\mu}
  \def\upi{\pi}
  \def\upartial{\partial}
\fi
\end{verbatim}
Wether you have AMS fonts or not, the \verb|\if..| and \verb|\fi| are
required in the above examples to ensure that when your article is
typeset in Monotype Times, the correct definitions for these symbols
are used.

The sample pages and guide can be made to use AMS fonts if you have them.
To use them, just uncomment the following line in the preamble
of \verb|mnguide.tex| and \verb|mnsample.tex|:
%
\begin{verbatim}
%\AMStwofontstrue
\end{verbatim}
%
If you do this, the following upright symbols are used in the
sample pages and guide: \verb|\upi|, \verb|\umu| and \verb|\upartial|.


\subsubsection{Special symbols}\label{SVsymbols}

The macros for the special symbols in Tables~\ref{mathmode} and~\ref{anymode}
have been taken from the Springer Verlag `Astronomy and Astrophysics' 
design, with their permission. They are directly compatible and use the 
same macro names.
These symbols will work in all text sizes, but are only guaranteed to work
in text and displaystyles. Some of the symbols will not get any smaller when
they are used in sub- or superscripts, and will therefore be displayed at the
wrong size. Don't worry about this as the typesetter will be able to sort
this out.
%
\begin{table*}
\begin{minipage}{110mm}
\caption{Special symbols which can only be used in math mode.}
\label{mathmode}
\begin{tabular}{@{}llllll}
Input & Explanation & Output & Input & Explanation & Output\\
\hline
\verb"\la"     & less or approx       & $\la$     &
  \verb"\ga"     & greater or approx    & $\ga$\\[2pt]
\verb"\getsto" & gets over to         & $\getsto$ &
  \verb"\cor"    & corresponds to       & $\cor$\\[2pt]
\verb"\lid"    & less or equal        & $\lid$    &
  \verb"\gid"    & greater or equal     & $\gid$\\[2pt]
\verb"\sol"    & similar over less    & $\sol$    &
  \verb"\sog"    & similar over greater & $\sog$\\[2pt]
\verb"\lse"    & less over simeq      & $\lse$    &
  \verb"\gse"    & greater over simeq   & $\gse$\\[2pt]
\verb"\grole"  & greater over less    & $\grole$  &
  \verb"\leogr"  & less over greater    & $\leogr$\\[2pt]
\verb"\loa"    & less over approx     & $\loa$    &
  \verb"\goa"    & greater over approx  & $\goa$\\
\hline
\end{tabular}
\end{minipage}
\end{table*}
%
\begin{table*}
\begin{minipage}{120mm}
\caption{Special symbols which don't have to be
used in math mode.}
\label{anymode}
\begin{tabular}{@{}llllll}
Input & Explanation & Output & Input & Explanation & Output\\
\hline
\verb"\sun"      & sun symbol            & $\sun$     &
  \verb"\degr"     & degree                & $\degr$\\[2pt]
\verb"\diameter" & diameter              & \diameter  &
  \verb"\sq"       & square                & \squareforqed\\[2pt]
\verb"\fd"       & fraction of day       & \fd        &
  \verb"\fh"       & fraction of hour      & \fh\\[2pt]
\verb"\fm"       & fraction of minute    & \fm        &
  \verb"\fs"       & fraction of second    & \fs\\[2pt]
\verb"\fdg"      & fraction of degree    & \fdg       &
  \verb"\fp"       & fraction of period    & \fp\\[2pt]
\verb"\farcs"    & fraction of arcsecond & \farcs     &
  \verb"\farcm"    & fraction of arcmin    & \farcm\\[2pt]
\verb"\arcsec"   & arcsecond             & \arcsec    &
  \verb"\arcmin"   & arcminute             & \arcmin\\
\hline
\end{tabular}
\end{minipage}
\end{table*}

\subsection{Bibliography}

References to published literature should be quoted in text by author
and date: e.g. Draine (1978) or (Begelman, Blandford \& Rees 1984).
Where more than one reference is cited having the same author(s) and date,
the letters a,b,c, \ldots\ should follow the date; e.g.\ Smith (1988a),
Smith (1988b), etc. 

\subsubsection{References in the text}

References in the text are given by author and date, and, whichever
method is used to produce the bibliography, the references in the text
are done in the same way. Each bibliographical entry has a key, which
is assigned by the author and used to refer to that entry in the text.
There is one form of citation -- \verb"\cite{key}" -- to produce the
author and date, and another form -- \verb"\shortcite{key}" -- which
produces the date only. Thus, Stella \& Campana \shortcite{sc} is
produced by
%
\begin{verbatim}
Stella \& Campana \shortcite{sc}
\end{verbatim}
%
while \cite{mtw} is produced by
%
\begin{verbatim}
\cite{mtw}
\end{verbatim}
%
When you introduce a three-author paper, you should list all
three authors at the first citation, and thereafter use et al.

\subsubsection{The list of references}

The following listing shows some references prepared in the style of
the journal; the code produces the references at the end of this guide.
The following rules apply for the ordering of your references:
%
\begin{enumerate}
  \item if an author has written several papers, some with other authors, 
        the rule is that the single-author papers precede the two-author
        papers, which, in turn, precede the multi-author papers;
  \item within the two-author paper citations, the order is determined
        by the second author's surname, regardless of date;
  \item within the multi-author paper citations, the order is
        chronological, regardless of authors' surnames.
\end{enumerate}
%
\begin{verbatim}
\begin{thebibliography}{}
  \bibitem[\protect\citename{Butcher }1992]{bu}
    Butcher J., 1992, Copy-editing: The Cambridge 
    Handbook, 3rd edn. Cambridge Univ. Press, 
    Cambridge
  \bibitem[\protect\citename{The Chicago Manual }%
    1982]{ch} The Chicago Manual of Style, 1982.
    Univ. Chicago Press, Chicago
  \bibitem[\protect\citename{Blanco }1991]{bl}
    Blanco P., 1991, PhD thesis, Edinburgh
    University
  \bibitem[\protect\citename{Brown \& Jones }%
    1989]{bj} Brown A. B., Jones C. D., 1989, 
    in Robinson E. F., Smith G. H., eds,
    Proc. IAU Symp. 345, Black Dwarfs. 
    Kluwer, Dordrecht, p. 210
  \bibitem[\protect\citename{Edelson }1987]{ed}
    Edelson R. A., 1987, ApJ, 313, 651
  \bibitem[\protect\citename{Lamport }1986]{la}
    Lamport L., 1986, \LaTeX: A Document 
    Preparation System. Addison--Wesley, New York
  \bibitem[\protect\citename{Mirabel \& Sanders }%
    1989]{ms} Mirabel I. F., Sanders D. B., 1989, 
    ApJ, 340, L53 
  \bibitem[\protect\citename{Misner et al.\ }%
    1973]{mtw} Misner C. W., Thorne K. S., 
    Wheeler J. A., 1973, Gravitation. 
    Freeman, San Francisco
  \bibitem[\protect\citename{Sopp \& Alexander }%
    1991]{sa} Sopp H. M., Alexander P., 1991, 
    MNRAS, 251, 112
  \bibitem[\protect\citename{Stella \& Campana }%
    1991]{sc} Stella L., Campana S., 1991, in 
    Treves A., Perola G. C., Stella L., eds, 
    Iron Line Diagnostic in X-ray Sources. 
    Springer--Verlag, Berlin, p. 230
\end{thebibliography}
\end{verbatim}
%
Each entry takes the form
%
\begin{verbatim}
\bibitem[\protect\citename{Author(s), }%
  Date]{tag} Bibliography entry
\end{verbatim}
%
where \verb"Author(s)" should be the author names as they are cited in
the text, \verb"Date" is the date to be cited in the text, and
\verb"tag" is the tag that is to be used as an argument for the
\verb"\cite{}" and \verb"\shortcite{}" commands. \verb"Bibliography
entry" should be the material that is to appear in the bibliography,
suitably formatted.

\subsection{Appendices}

The appendices in this guide were generated by typing:
%
\begin{verbatim}
\appendix
\section{For authors}
     :
\section{For editors}
\end{verbatim}
%
You only need to type \verb"\appendix" once. Thereafter, every
\verb"\section" command will generate a new appendix which will be 
numbered A, B, etc. 


\section[]{Example of section heading with\\*
  S{\sevensize\bf MALL} C{\sevensize\bf APS},
  \lowercase{lowercase}, \textbfit{italic},
  and bold\\* Greek such as
  $\bmu^{\bkappa}$}\label{headings}

\ifoldfss
%
There are at least two ways of achieving this section head. The first
involves the use of \verb"\boldmath". You could say:
%
\begin{verbatim}
\section[]{Example of section heading with\\*
  S{\sevensize\bf MALL} C{\sevensize\bf APS},
  \lowercase{lowercase}, \textbfit{italic},
  and bold\\* Greek such as
  \mbox{\boldmath{$\mu^{\kappa}$}}}
\end{verbatim}
%
Many implementations of \LaTeX\ do not support \verb"\boldmath" at 9pt,
so you may need to use the bold Greek characters as described in
Section~\ref{boldgreek}, and typeset the section head as follows:
%
\begin{verbatim}
\section[]{Example of section heading with\\*
  S{\sevensize\bf MALL} C{\sevensize\bf APS},
  \lowercase{lowercase}, \textbfit{italic},
  and bold\\* Greek such as
  $\bmu^{\bkappa}$}
\end{verbatim}
%
\else % nfss
%
Was produced with:
%
\begin{verbatim}
\section[]{Example of section heading with\\*
  S{\sevensize MALL} C{\sevensize APS},
  \lowercase{lowercase}, \textbfit{italic},
  and bold\\* Greek such as
  $\bmu^{\bkappa}$}
\end{verbatim}
%
\fi

\begin{thebibliography}{}
  \bibitem[\protect\citename{Butcher }1992]{bu}
    Butcher J., 1992, Copy-editing: The Cambridge 
    Handbook, 3rd edn. Cambridge Univ. Press, 
    Cambridge
  \bibitem[\protect\citename{The Chicago Manual }%
    1982]{ch} The Chicago Manual of Style 1982. Univ.
    Chicago Press, Chicago
  \bibitem[\protect\citename{Blanco }1991]{bl}
    Blanco P., 1991, PhD thesis, Edinburgh 
    University
  \bibitem[\protect\citename{Brown \& Jones }%
    1989]{bj} Brown A. B., Jones C. D., 1989, 
    in Robinson E. F., Smith G. H., eds,
    Proc. IAU Symp. 345, Black Dwarfs. 
    Kluwer, Dordrecht, p. 210
  \bibitem[\protect\citename{Edelson }1987]{ed}
    Edelson R. A., 1987, ApJ, 313, 651
  \bibitem[\protect\citename{Lamport }1986]{la}
    Lamport L., 1986, \LaTeX: A Document 
    Preparation System. Addison--Wesley, New York
  \bibitem[\protect\citename{Mirabel \& Sanders }%
    1989]{ms} Mirabel I. F., Sanders D. B., 1989, 
    ApJ, 340, L53 
  \bibitem[\protect\citename{Misner et al.\ }%
    1973]{mtw} Misner C. W., Thorne K. S., 
    Wheeler J. A., 1973, Gravitation. 
    Freeman, San Francisco
  \bibitem[\protect\citename{Sopp \& Alexander }%
    1991]{sa} Sopp H. M., Alexander P., 1991, 
    MNRAS, 251, 112
  \bibitem[\protect\citename{Stella \& Campana }%
    1991]{sc} Stella L., Campana S., 1991, in 
    Treves A., Perola G. C., Stella L., eds, 
    Iron Line Diagnostic in X-ray Sources. 
    Springer--Verlag, Berlin, p. 230
\end{thebibliography}


\appendix
\section{For authors}

Table~\ref{authors} is a list of design macros which are unique to MN. The
list displays each macro's name and description.

\begin{table*}
\begin{minipage}{155mm}
\caption{Authors' notes.}
\label{authors}
\begin{tabular}{@{}ll}
\verb"\title[optional short title]{long title}"
                    & short title used in running head\\
\verb"\author[optional short author(s)]{long author(s)}"
                    & short author(s) used in running head\\
\verb"\newauthor"   & starts a new line in the author environment\\
\verb"\begin{abstract}...\end{abstract}"& for abstract on titlepage\\
\verb"\begin{keywords}...\end{keywords}"& for keywords on titlepage\\
\verb"\nokeywords"  & if there are no keywords on titlepage\\
\verb"\begin{figure*}...\end{figure*}" & for a double spanning figure in two-column mode\\ 
\verb"\begin{table*}...\end{table*}" & for a double spanning table in 
                                       two-column mode\\ 
\verb"\plate{Opposite p.~812, MNRAS, {\bf 261}}"
                  & used with \verb"\thispagestyle{plate}" for plate pages\\
\verb"\contcaption{}" & for continuation figure and table captions\\
\verb"\bmath{math text}" & Bold math italic / symbols.\\
\verb"\textbfit{text}", \verb"\mathbfit{text}" & Bold text italic
   (defined in the preamble of \verb"mnsample.tex").\\
\verb"\textbfss{text}", \verb"\mathbfss{text}" & Bold text sans serif
   (defined in the preamble of \verb"mnsample.tex").\\
\end{tabular}
\end{minipage}
\end{table*}


\section{For editors}

The additional features shown in Table~\ref{editors} may be used for
production purposes.

\begin{table*}
\begin{minipage}{155mm}
\caption{Editors' notes.}
\label{editors}
\begin{tabular}{@{}lp{270pt}}
\verb"\pagerange{000--000}"& for catchline, note use of en-rule\\
\verb"\pagerange{L00--L00}"& for letters option, used in catchline\\
\verb"\volume{000}" & volume number, for catchline\\
\verb"\pubyear{0000}" & publication year, for catchline\\
\verb"\microfiche{MN000/0}" & for articles accompanied by microfiche\\
\verb"\journal" & replace the whole catchline at one go\\
\verb"[doublespacing]" & documentstyle option for doublespacing\\
\verb"[galley]" & documentstyle option for running to galley\\
\verb"[landscape]" & documentstyle option for landscape illustrations\\
\verb"[letters]" & documentstyle option, for short communications 
                   (adds L to folios)\\
\verb"[onecolumn]" & documentstyle option for one-column \\
\verb"[referee]" & documentstyle option for 12/20pt, single col, 
                   39pc measure\\
\verb"\bsp" & typesets the final phrase `This paper has been produced
              using the\\
            & Royal Astronomical Society/Blackwell Science \LaTeX\ style
              file.'\\
\end{tabular}
\end{minipage}
\end{table*}


%%%%%%%%%%%%%%%%%%%%%%%%%%%%%%%%%%%%

\makeatletter
% define \thebiblio (same as thebibliography, but
% without the section heading)
\def\thebiblio#1{%
 \list{}{\usecounter{dummy}%
         \labelwidth\z@
         \leftmargin 1.5em
         \itemsep \z@
         \itemindent-\leftmargin}
 \reset@font\small
 \parindent\z@
 \parskip\z@ plus .1pt\relax
 \def\newblock{\hskip .11em plus .33em minus .07em}
 \sloppy\clubpenalty4000\widowpenalty4000
 \sfcode`\.=1000\relax
}
\let\endthebiblio=\endlist
\makeatother


\section{Monthly Notices journal style}
        
Authors  submitting  \TeX\ or \LaTeX\ papers to Monthly  Notices  
may wish to note the following points regarding journal style. 
Adherence to correct style from the start will obviously save time
and effort later on, in terms of fewer requested subeditorial 
corrections.  The notes given here relate to common style errors found 
in  Monthly Notices manuscripts, and are {\it not\/} intended to  be  
exhaustive. Please see the editorials in issues 257/2 and 260/1, as 
well as any recent issue of the journal, for more details. As far 
as  possible, the subeditor will indicate which of  your  changes 
would be best done globally, thus saving you time.
        
\subsection{Punctuation}

When  deciding  where to add commas, it may be  helpful  to  read 
through  the sentence and note where the natural `pauses'  occur. 
The  needs  of readers for whom English is not a  first  language 
should  be  borne in mind when punctuating  long  sentences.  For 
example, consider the following sentence as it appeared in Monthly Notices:
`When we do not limit ourselves by constraints arising 
from the choice of an initial fluctuation spectrum, structures in 
an open universe, including the peculiar velocity structure,  can 
be  reproduced  in a flat Lema\^{\i}tre universe for a large  part  of 
their evolution.' Now consider the same sentence without  commas: 
`When  we do not limit ourselves by constraints arising from  the 
choice  of an initial fluctuation spectrum structures in an  open 
universe including the peculiar velocity structure can be  reproduced  
in  a  flat Lema\^{\i}tre universe for a large  part  of  their 
evolution.'
        
\subsection{Spelling}

Please  use British spelling -- e.g.\ centre not center,  labelled 
not  labeled. The following style regarding -ise and -ize  spellings  is
followed.  -ise: devise,  surprise,  comprise,  revise, 
exercise,  analyse. -ize: recognize, criticize, minimize,  emphasize, organize.
        
\subsection{Titles and section/subsection headings}

With the exception of section headings (e.g.\
{\bf INTRO\-DUC\-TION}\ldots),
capital letters should be used only where they occur in a normal
sentence -- e.g.\ 
`\textbfit{ROSAT\/} {\bf observations of the unusual star}\ldots', not
`\textbfit{ROSAT\/} {\bf Observations of the Unusual Star}\ldots'.

\subsection{Key words}

The  list of Monthly Notices key words is published with the  4th 
issue  of each volume. No other key words should be used.  Please 
use the correct layout for key words: 
\smallskip        

\noindent {\bf Key words:} galaxies: active -- galaxies: Seyfert -- 
radio continuum: galaxies.
        
\subsection{Hyphens and N-rules}

\begin{enumerate}
\item Hyphens (one dash in \TeX/\LaTeX). Monthly Notices uses hyphens 
for compound adjectives (e.g.\ low-density gas, least-squares fit, 
two-component  model).  This also applies to simple  units  (e.g.\ 
1.5-m  telescope,  284.5-nm line), but not to  complex  units  or 
ranges,  which could become cumbersome (e.g.\ 15 km  s$^{-1}$  feature, 
100--200 $\umu$m observations).

\item N-rules (two dashes in \TeX/\LaTeX). These are used  (a)  to 
separate  key  words, (b) as parentheses (e.g.\  `the  results  -- 
assuming no temperature gradient -- are indicative of\ldots), (c) to 
denote a range (e.g.\ 1.6--2.2 $\umu$m), and (d) to denote the  joining 
of   two  words  (e.g.\  Kolmogorov--Smirnov  test,   Herbig--Haro 
object).
        
\item The  M-rule (three dashes in \TeX/\LaTeX) is  not  used  in 
Monthly Notices.
\end{enumerate}

        
\subsection{References}

It  is important to use the correct reference style,  details  of 
which  can be found in the editorials cited above, and  which  is 
demonstrated in any recent issue of the journal. The main  points 
are summarized below.
%        
\begin{enumerate}
\item In text -- for three-author papers, give all three  authors 
at first mention, and `et al.' thereafter.

\item In references list -- no bold or italic, no  commas  after 
author surnames, and no ampersand between final two author names.
Use  simplified abbreviations for frequently used journals  (e.g.\ 
MNRAS,  ApJ,  A\&A, PASP -- see full list in  editorial  in  issue 
260/1). List all authors if eight or less, otherwise `et al.' For 
example,
\end{enumerate}        

\begin{thebiblio}{}% you should use {thebibliography} here.
\bibitem{} Biggs J.D., Lyne A.G., 1992, MNRAS, 254, 257

\bibitem{} Brown A.B., Jones C.D., 1989, in Robinson E.F., 
Smith G.H.,  eds, Proc. IAU Symp. 345, Black Dwarfs. Kluwer, Dordrecht, 
p.~210

\bibitem{} Felsteiner  J.,  Opher  R., 1991, in Treves A.,  ed.,  
Iron  Line Diagnostics in X-ray Sources. Springer-Verlag, Berlin, p.~209

\bibitem{} Gunn  J.E., Knapp G., 1993, in Soifer B.T., ed., ASP  
Conf.\ Ser.\ Vol.~43, Sky Surveys. Astron.\ Soc.\ Pac., San Francisco, p.~267

\bibitem{} Peebles P.J.E., 1980, The Large-Scale Structure of the  Universe. Princeton Univ. Press, Princeton, NJ 

\bibitem{} Pounds K.A. et al., 1993, MNRAS, 260, 77

\bibitem{} Williams B.G., 1992, PhD thesis, Univ.\ Edinburgh 
\end{thebiblio}

\subsection{Maths}
Scalar  variables are italic; vectors are bold  italic;  matrices 
are  `bold Univers' font (like bold sans serif). Differential  d, 
complex  i, exponential e, sin, cos, tan, log, etc.,  are  roman. 
Sub/superscripts  that are physical variables are  italic,  while 
those  that are just labels are roman (e.g.\ $C_p$, but $T_{\rmn{eff}}$).
Equations should be punctuated as part of the sentence.
        
\subsection{Miscellaneous}

e.g., i.e., cf., etc., are roman.
Single quotes `  ' not double quotes `` ".
Take care to use correct units -- see recent \hbox{issues} for  details. 
Use  superscript $-1$, not solidus /, for units -- e.g.\ km~s$^{-1}$ not 
km/s. The unit of arcseconds is arcsec when used to denote  angular  
size  or separation (e.g.\ `beamsize 12~arcsec',  `30~arcsec 
west  of  the star'); use \arcsec\ for positions (e.g.\  
Dec.~$-30^\circ$~29\arcmin~23\arcsec) (similarly for arcminutes). 
The unit of magnitudes is  mag, 
not superscript m. Percentages should be written `per cent',  not 
\%.  Use the degree symbol $^\circ$ (\verb"$^\circ$")  except to denote, e.g., areas,  where      
`deg$^2$'  may be more appropriate (e.g.\ `a survey area of 3~deg$^2$'). 
Degree symbols should be positioned above the decimal  point  if 
there  is one -- i.e.\ 23\fdg 4 not 23.4$^\circ$ (similarly for  superscript 
h, m, s, and \arcmin, \arcsec\ symbols in coordinates). See guide for coding.
Ionized  species  should be denoted by small caps,  with  a  thin 
space -- e.g.\ \hbox{He\,{\sc ii}}, \hbox{C\,{\sc iv}}, 
\hbox{[Fe\,{\sc ii}]} 465.8 nm, \hbox{N\,{\sc iii}} 463.4~nm.  If 
lack  of the correct font at your site prevents this from  coming 
out on your printout, it would be helpful if you were to indicate 
as such on your manuscript so that the subeditor knows that there 
is no need for correction.
Computer software (e.g.\ {\sc figaro}) should be in small capitals.
Satellite names should be italic (e.g.\ {\it Ginga, IRAS\/}).
The correct bracket order is \{[( )]\}.

%%%%%%%%%%%%%%%%%%%%%%%%%%%%%%%%%%%%

\section{Trouble-shooting}

Authors may from time to time encounter problems with the  preparation  
of their papers in \TeX/\LaTeX. The appropriate  action  to 
take will depend on the nature of the problem -- the following is 
intended to act as a guide.
%
\begin{enumerate}        
\item If a problem is with \TeX/\LaTeX\ itself, rather than with the 
actual macros, please refer to the appropriate handbooks for 
initial advice.\footnote{\TeX : Knuth D., 1986, The \TeX book. 
Addison Wesley; \LaTeX: Lamport L., 1985, \LaTeX\ User's Guide and 
Reference Manual. Addison Wesley.} If the 
solution cannot be found, and you suspect that the problem lies 
with the macros, then please contact the RAS Journal Production
team at Blackwell Science (BS), 23 Ainslie
Place, Edinburgh EH3 6AJ, UK (Tel: 031 226 7232; Fax: 031 226 3803;
omit the first zero if calling from outside the UK).
The BS office will shortly be
accessible by email -- please see the Instructions for Authors on 
the inside back cover of Monthly Notices for details. Please 
provide precise details of the problem (what you were trying to 
do -- ideally, include examples of source code as well -- and 
what exactly happened; what error message was received).
        
\item Problems with page make-up, particularly in the two-column 
mode (e.g.\ large spaces between paragraphs, or under headings or 
figures; uneven columns; figures/tables appearing out of order). 
Please do {\it not\/} attempt to remedy these yourself using `hard' page 
make-up commands -- the typesetters at Cambridge University Press 
(CUP) will sort out problems when typesetting. (You may, if you 
wish, draw attention to particular problems when submitting the 
final version of your paper.)
        
\item If a required font is not available at your site, allow \TeX\
to substitute the font and report the problem on your disk 
documentation form. 

\item If you choose to use \verb"\boldmath", you may find that boldmath has not been defined locally for use with a particular size of font. If this is the case, you will get a message that reads something like:
%
\begin{verbatim}
LaTeX Warning: No \boldmath typeface in this size, 
using \unboldmath on input line 44.
\end{verbatim}

If you get this message, you are advised to use the alternative described
in this guide for attaining bold face math italic characters,
i.e.\ \verb"\bmath{...}".
\end{enumerate}
        
        
\subsection{Fixes for coding problems}

The new versions of the style files and macros have been designed
to  minimize the need for user-defined macros to  create  special
symbols. Authors are urged, wherever possible, to use the following
coding rather than create their own. This will minimize  the
danger of author-defined macros being accidentally  `over-ridden'
when the paper is typeset in Monotype Times (see Section~\ref{TMth},
`Typesetting  mathematics', in the \LaTeX\ author guide).
%
\begin{enumerate}
\item Fonts in sections and paper titles. The following are  examples
of styles that sometimes prove difficult to code.
\end{enumerate}


\subsubsection*{P\lowercase{aper titles}}

\boxit{\huge\bf
  A survey of \textbfit{IRAS\/} galaxies at
  $\bmath{\delta > \bld{50}^\circ}$}
%
is produced by:
%
\begin{verbatim}
\title[A survey of IRAS galaxies at
       $\delta > 50^\circ$]
  {A survey of \textbfit{IRAS\/} galaxies at
   $\bmath{\delta > \bld{50}^\circ}$}
\end{verbatim}
\bigskip

\boxit{\huge\bf Observations of compact H\,{\Large\bf II} regions}
%
is produced by:
%
\begin{verbatim}
\title[Observations of compact H\,{\normalsize
       \it II} regions]
  {Observations of compact H\,{\Large\bf II}
   regions}
\end{verbatim}


\subsubsection*{S\lowercase{ection headings}}

\boxit{\bf 1\quad THE \textbfit{IRAS\/} DATA FOR
  $\bmath{\delta > \bld{50}^\circ}$}
%
is produced by:
%
\begin{verbatim}
\section[]{The \textbfit{IRAS\/} data for
  $\bmath{\delta > \bld{50}^\circ}$
\end{verbatim}
\bigskip

\boxit{\bf 2\quad H\,{\sevensize\bf II} GALAXIES AT
  $\bmath{\lowercase{z} > \bld{1.6}}$}
%
is produced by:
%
\begin{verbatim}
\section[]{H\,{\sevensize\bf II} galaxies at
  $\bmath{\lowercase{z} > \bld{1.6}}$}
\end{verbatim}


\subsubsection*{S\lowercase{ubsection headings}}
        
\boxit{\bf 2.1\quad The \textbfit{IRAS\/} data for
  $\bmath{\delta > \bld{50}^\circ}$: galaxies\\
  at $\bmath{z > \bld{1.5}}$}
%
is produced by:
%
\begin{verbatim}
\subsection[]{The \textbfit{IRAS\/} data for
  $\bmath{\delta > \bld{50}^\circ}$: galaxies\\
  at $\bmath{z > \bld{1.5}}$}
\end{verbatim}
\bigskip

\boxit{\bf 2.2\quad Observations of compact H\,{\sevensize\bf II} regions}
%
is produced by:
%
\begin{verbatim}
\subsection[]{Observations of compact 
  H\,{\sevensize\bf II} regions}
\end{verbatim}
\bigskip

\boxit{\it 2.2.1\quad A survey of radio galaxies for
  $\delta > \itl{50}^\circ$}
%
is produced by:
%
\begin{verbatim}
\subsubsection[]{A survey of radio galaxies for 
  $\delta > \itl{50}^\circ$}
\end{verbatim}
\bigskip
        
\boxit{\it 2.2.2\quad Determination of T$_{eff}$ in compact
  H\,{\sevensize\it II} regions}
%
is produced by:
%
\begin{verbatim}
\subsubsection[]{Determination of T$_{eff}$ in
  compact H\,{\sevensize\it II} regions}
\end{verbatim}
\bigskip

\begin{enumerate}
\stepcounter{enumi}

\item Small capitals and other unusual fonts in table and figure captions:
\par\smallskip
\boxit{\small {\bf Figure 1.} Profiles of the H$\alpha$ and
  N\,{\sc iii} lines observed.}
%
is produced by:
%
\begin{verbatim}
\caption{Profiles of the H$\alpha$ and
  N\,{\sc iii} lines observed.}
\end{verbatim}
        
\item Multiple author lists (to get the correct vertical spacing 
and wraparound on the title page of a multiple-author paper).
\par\smallskip

\boxit{\huge\bf
  The variation in the\newline 
  strength of low-$\bmath{l}$ solar 
  $\bmath{p}$-modes: 1981--2
\medskip

\LARGE
  Y. Elsworth, R. Howe, G.R. Isaac, C.P. McLeod,  
  B.A. Miller, R. New,
  C.C. Speake and S.J. Wheeler}
%
is produced by:
%
\begin{verbatim}      
\title[The variation in the strength of low-$l$ 
  solar $p$-modes: 1981--2]%
  {The variation in the strength of 
  low-$\bmath{l}$ solar 
  $\bmath{p}$-modes: 1981--2}
        
\author[Y. Elsworth et al.]
  {Y. Elsworth, R. Howe, G.R. Isaac,\cr 
  C.P. McLeod, B.A. Miller, R. New,\cr 
  C.C. Speake and S.J. Wheeler}
\end{verbatim}        
        
\item Ionized species (as used in the examples above). The correct 
style calls for the use of small capitals and a thin space after 
the symbol for the element: e.g.\ for \hbox{H\,{\sc ii}}, use the code 
\verb"\hbox{H\,{\sc ii}}". The use of the \verb"\hbox" will stop the 
H and the {\sc ii} being separated.

\item Lower case greek pi ($\pi$), mu ($\mu$) and partial ($\partial$).
In certain circumstances, the Monthly Notices style calls for these to be
roman [when pi is used to denote the constant 3.1415$\ldots$, mu is
used to denote `micro' in a unit (e.g.\ $\umu$m, $\umu$Jy), and partial
is a differential symbol]. See Subsubsection~\ref{upgreek} for instructions.

\item Decimal degrees, arcmin, arcsec, hours, minutes and seconds. 
The symbol needs to be placed vertically above the decimal point. 
For example, the sentence
%
\begin{quote}
The observations were made along position angle
120\fdg 5, starting from the central coordinates
$\rmn{RA}(1950)=19^{\rmn{h}}~22^{\rmn{m}}~18\fs2$,
$\rmn{Dec.}~(1950)=45^\circ~18'~36\farcs 4$
\end{quote}
%        
uses the following coding:
%
\begin{verbatim}
The observations were made along position angle
120\fdg 5, starting from the central coordinates
$\rmn{RA}(1950)=19^{\rmn{h}}~22^{\rmn{m}}~18\fs2$,
$\rmn{Dec.}~(1950)=45^\circ~18'~36\farcs 4$
\end{verbatim}

\item The correct coding for the prime symbol \arcmin\ is 
\verb"\arcmin", and that for \arcsec\ is \verb"\arcsec"; see the two
tables on special symbols.
        
\item N-rules, hyphens and minus signs (see style guide for  
correct usage). To create the correct symbols in the sentence
%        
\begin{quote}
The high-resolution observations were made along a
line at an angle of $-15^\circ$ (east from north)
from the axis of the jet, which runs north--south
\end{quote}
you would use the following code:
%
\begin{verbatim}
The high-resolution observations were made along a
line at an angle of $-15^\circ$ (east from north)
from the axis of the jet, which runs north--south
\end{verbatim}

\item Vectors and matrices should be bold italic and bold sans
serif respectively. To create the correct fonts for the vector $\bmath{x}$
and the matrix \textbfss{P}, you should use \verb"$\bmath{x}$" and
\verb"\textbfss{P}" respectively; \verb"\mathbfss" is for use in
math mode. Bold face text italic can be obtained by using
\verb"\textbfit{..}" and \verb"\mathbfit{..}" for math mode.
        
\item Bold italic superscripts and subscripts. To get  these 
to  come  out  in the correct font and the right  size,  
you need to use \verb"\bmath". You can create the output
$\bmath{k_x}$ by typing \verb"$\bmath{k_x}$".
Try to avoid using \LaTeX\ commands to determine script sizes 
that are already  defined in the style file. For example, macros such as
%
\begin{verbatim}        
\newcommand{\th}{^\mbox{\tiny th}}
\end{verbatim}
%        
are generating extra work;
%
\begin{verbatim}
\newcommand{\th}{^{th}}
\end{verbatim}
%
will do, and  will get  the  size  of the superscript right whether  
in  main  text, tables or captions (the use of \verb"\tiny" over-rides 
the style  file). 
Also, the \verb"\mbox" is not necessary, as \TeX\ won't split a 
superscript/subscript from its variable at a line break.

\item Calligraphic letters (uppercase only).
%
\ifnfsstwo
%
Normal uppercase calligraphic can be produced with \verb"\mathcal" as
normal (in math mode). Bold calligraphic can be produced with \verb"\bmath".
e.g.\ \verb"$\bmath{\mathcal A}$" gives $\bmath{\mathcal A}$.
%
\else
%
Normal uppercase calligraphic can be produced with \verb"\cal" as
normal (in math mode). Bold calligraphic can be produced with \verb"\bmath".
e.g.\ \verb"$\bmath{\cal A}$" gives $\bmath{\cal A}$.
%
\fi

\item Automatic scaling of brackets. The codes \verb"\left" and  
\verb"\right" should  be used to scale brackets automatically to
fit the equation being set. For example, to get
\[
  v = x \left( \frac{N+2}{N} \right)
\]        
use the code
%
\begin{verbatim}
\[
  v = x \left( \frac{N+2}{N} \right)
\]        
\end{verbatim}        
        
\item Roman font in equations. It is often necessary to make some 
symbols roman in an equation (e.g.\ units, subscripts). For  example, 
to get the following output:
\[
  \sigma \simeq (r/13~h^{-1}~\rmn{Mpc})^{-0.9}, 
  \qquad \omega = \frac{N-N_{\rmn{s}}}{N_{\rmn{R}}},
\]
you should use:
%
\begin{verbatim}
\[
  \sigma \simeq (r/13~h^{-1}~\rmn{Mpc})^{-0.9}, 
  \qquad \omega=\frac{N-N_{\rmn{s}}}{N_{\rmn{R}}},
\]
\end{verbatim}

\item Continuation figure and table captions.
See Section~\ref{contfigtab}.
\end{enumerate}


\subsection{Springer-Verlag macro names}

These have been incorporated from the Astronomy \& Astrophysics \LaTeX\ 
style file, to aid in the creation of various commonly used
astronomical symbols. Please see Subsection~\ref{SVsymbols} for details.

%%%%%%%%%%%%%%%%%%%%%%%%%%%%%%%%%%%%

\bsp % ``This paper has been produced using the ...''

\label{lastpage}

\end{document}
